%%This is a very basic article template.
%%There is just one section and two subsections.
\documentclass{article}

\begin{document}


\section{Validity-Guided Synthesis from Assume-Guarantee Contracts}

\subsection{Preliminaries}
(Text from TACAS submission)
We describe a system using the disjoint sets $state$ and $inputs$.
Formally, an \emph{implementation} is a \emph{transition system}
described by an initial state predicate $I(s)$ of type $state \to
bool$ and by a transition relation $T(s,i,s')$ of type $state \to
inputs \to state \to bool$.

An Assume-Guarantee (AG) contract can be formally defined by a set of
\emph{assumptions} and a set of \emph{guarantees}. The
\emph{assumptions}, $A: state \rightarrow inputs \rightarrow bool$,
impose constraints over the inputs which may be modal in terms of the
previous state. The \emph{guarantees} $G$ consist of two separate
subsets $G_I: state \rightarrow bool$ and $G_T: state \rightarrow
inputs \rightarrow state \rightarrow bool$, where $G_I$ defines the
set of valid initial states, and $G_T$ specifies the properties that
need to be met during each new transition between two states. Note
that we do not necessarily expect that a contract would be defined
over all variables in the transition system, but we do not make any
distinction between internal state variables and outputs in the
formalism. This way, we can use state variables to (in some cases)
simplify specification of guarantees.

\subsection{Approach}

The main idea of the algorithm is to use over- and underapproximations to
create a set of states, say $F(s)$ in which the system can safely use the
transition relation without violating the contract. Due to AE-VAL's machinery,
we are able to finally extract a Skolem Function $S_k$ that describes an
implementation. The algorithm is split into two different phases. The first is
focused on finding ``good'' initial states.

Initially, $F(s) = true$. We then ask AE-VAL whether the formula $\forall s,i.
F(s) \land A(s,i) \land G_{I}(s) \Rightarrow \exists s'. G_{T}(s,i,s') \land
F(s')$.
If the formula is valid, any state can be initial with the given contract. In
this case, we proceed with extracting a Skolem function that describes a
transition to a new ``good'' state.

Alternatively, AE-VAL will return with a ``non-valid'' answer. In addition to
this, a subset, named $Q(s,i)$ is also generated, for which the formula
$\forall s,i. Q(s,i) \Rightarrow \exists s'. G_{T}(s,i,s')$ is valid.
We can use this subset to refine $F(s)$ in the following way:

\begin{itemize}
  \item Define $R(s) = true$. Examine the validity of $\forall s. R(s)
  \Rightarrow \exists i. \lnot Q(s,i)$. If the formula is valid, the
  contract cannot be realizable, as for every state $s$, there are inputs for
  which the assumptions are violated. On the other hand, if AE-VAL returns ``non-valid'' we receive
  a valid subset of $R(s)$, namely $W(s)$.
  \item By definition, we now have that $\forall s. W(s) \Rightarrow \exists i.
  \lnot Q(s,i)$. The set $W(s)$ effectively describes a region of states for
  which the contract is violated, and therefore will have to be blocked from the
  resulting set that describes the system's ``good'' states. Thus, we refine
  $F(s) \equiv F(s) \land \lnot W(s)$.
  \item We plug the refined $F(s)$ in the original formula, and reiterate the
  process. Eventually, a point is reached where we get a valid answer and
  extract a Skolem relation that precisely describes $F(s) \land A(s,i) \land
  G_I{s}$.
\end{itemize}

The second phase of the algorithm is identical to the first, but does not have
the requirement that a state satisfies $G_I{s}$. Thus, we begin by checking the
validity of $\forall s,i. F(s) \land A(s,i) \Rightarrow \exists s'
G_{T}(s,i,s') \land F(s')$. The process to refine $F(s)$ is identical to the
previous phase. Again, we eventually either decide that the contract is
unrealizable, or are able to extract a Skolem function that describes a
transition between states that comply to the contract.


\subsection{Notes and Comments}

A possible way to optimize the algorithm's convergence described above, would be
to replace $F(s) = true$ with a refined set $F'(s)$ resulting from the recursive
blocking of bad states, starting from the formula $\forall s,i. A(s,i) \land G_I(s) \Rightarrow
\exists s'. G_T(s,i,s')$. Then, the algorithm would execute the first phase by
checking the validity of $\forall s,i. F'(s) \Rightarrow exists s'.
G_{T}(s,i,s') \land F'(s')$. A similar approach can be also applied to the
second phase.
\end{document}
